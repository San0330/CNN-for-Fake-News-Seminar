\chapter{Introduction}

\section{Introduction}

Fake news refers to misinformation or false information presented as news with the intent to deceive. In the digital age, online platforms have enabled both journalists and non-journalists to reach a massive audience, disseminating information that includes articles, claims, statements, posts, and other types of content related to public figures. This democratization of information sharing raises significant social concerns regarding the authenticity and intentions behind the news. Fake news can have severe real-world consequences, such as inciting violence, manipulating stock markets, and exacerbating political issues.\\

Detecting fake news is a critical application of \ac{nlp}. Automated fake news detection is the task of assessing the truthfulness of claims in news.\cite{oshikawa2020survey} To prevent the spread of fake news, it is essential to adopt effective strategies. Deep learning techniques, in particular, show great promise in the task of fake news detection. This paper explores the use of Convolutional Neural Network (CNN) and its variants to classify news articles. The primary objective of fake news detection is to accurately identify and distinguish fake news from genuine news. \\

CNNs are widely used as they succeed in many text classification tasks. It is used for extracting features with a variety of metadata. CNN-LSTM is a variant that uses both \ac{cnn} and \ac{lstm} model, where CNN is used for feature extraction and LSTM for sequence modelling. Another powerful variant involves the integration of attention mechanisms with CNNs. The attention mechanism helps the model focus on the most crucial parts of the text, thereby improving the overall performance in text classification tasks. By selectively emphasizing significant features, the model can better understand the context and nuances within the text, leading to more accurate fake news detection. \\

The aim of this research is to compare various CNN-based models and analyze their performance in the task of fake news classification. We delve into the specifics of each variant, evaluating their strengths and limitations. By conducting a comprehensive analysis, we aim to identify the most effective model for detecting fake news, contributing to the broader effort of combating misinformation.

\section{Problem Statement}

Spreading misinformation through fake news presents a challenge for human society. It causes negative impacts in social, political and economical sectors and is a major threat to democracy. Early detection of fake news possibly at its source can help to prevent damages that can be caused by it. In the age of social media, it's more difficult to tackle the spread and impact of fake news. An efficient algorithm needs to be explored so that it becomes difficult to spread fake news. Identifying the vocabulary that is used to mislead readers is the essential task of identifying fake news. A difficult challenge is classifying news through word-level context. As a result, the purpose of this research is to use hidden patterns in news text to detect fake news.

\section{Objective}

\textbf{The objective of this seminar is}
\begin{itemize}
    \item To compare the efficiency and performance of three CNN architectures: Baseline CNN, CNN-LSTM, and CNN-Attention in the context of fake news detection, utilizing a comprehensive dataset.
\end{itemize}

\clearpage