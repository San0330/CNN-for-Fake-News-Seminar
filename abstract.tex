\centerline{\large\textbf{Abstract}}

\vspace{1cm}

The explosion of social media allowed individuals to spread information without cost, with little investigation and fewer filters than before. This amplified the problem of fake news, which has become a major concern nowadays due to the negative impact it brings to the communities. To tackle the rise and spread of fake news, automatic detection techniques have been researched building on artificial intelligence and machine learning. The recent achievements of deep learning techniques in complex natural language processing tasks, make them a promising solution for fake news detection too. This work researches a hybrid deep learning model that combines convolutional and recurrent neural networks and attention weights for fake news classification. The baseline CNN model and CNN-LSTM showed similar precision, recall, and F1-score metrics with 96\% on all metrics, while CNN-Attention has a slight increase in precision, recall, and F1-score with 97\% on all metrics. CNN model and CNN-LSTM showed an accuracy of about 96\% and CNN-Attention showed an accuracy of 97.14\%. The confusion matrix showed CNN-Attention has the highest true positives while baseline CNN has the highest true negative predictions. \\

\textbf{Keywords:}  CNN, CNN-LSTM, CNN-Attention, fake news detection

